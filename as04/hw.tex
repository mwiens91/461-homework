% Set up the document
\documentclass{article}

% Page size
\usepackage[
    letterpaper,]{geometry}

% Lines between paragraphs
\setlength{\parskip}{\baselineskip}
\setlength{\parindent}{0pt}

% Math
\usepackage{mathtools}
\usepackage{amssymb}
\usepackage{amsthm}
\usepackage{commath}

% Number sets
\newcommand{\C}{\mathbb{C}}
\newcommand{\N}{\mathbb{N}}
\newcommand{\Q}{\mathbb{Q}}
\newcommand{\R}{\mathbb{R}}
\newcommand{\Z}{\mathbb{Z}}

% Links
\usepackage{hyperref}

% Page numbers at top right
\usepackage{fancyhdr}
\pagestyle{fancy}
\fancyhf{}
\fancyhead[R]{\thepage}
\renewcommand\headrulewidth{0pt}

% Graphics
\usepackage{float}
\usepackage{graphicx}
\graphicspath{ {./img/} }

\begin{document}

\textbf{MATH 461 assignment 4} \\
\textbf{Matt Wiens \#301294492} \\
\textbf{2020-07-03}

3.9.2. Psychologists interested in learning theory study learning
curves. A learning curve is a graph of a function $P(t)$, the
performance of someone learning a skill as a function of the training
time $t$.

(a) What does $\od{P}{t}$ represent?

\textit{Solution.}

\vspace{5mm}

(b) Discuss why the differential equation
%
\begin{equation*}
    \dod{P}{t} = k (M - P),
\end{equation*}
%
where $k$ and $M$ are positive constants, is a reasonable model for
learning. What is the meaning of $k$ and $M$? What would be a reasonable
initial condition for the model? Include a graph of $\od{P}{t}$ versus
$P$ as part of your discussion.

\textit{Solution.}

\vspace{5mm}

(c) Make a qualitative sketch of solutions to the differential equation.

\textit{Solution.}

\newpage

3.9.16. Some populations produce waste products, which in high concentrations are
toxic to the population itself. For example, algae or bacteria show the structure
in Figure 3.25 (in the course textbook). Let the population density be denoted
by $n(t)$ and the toxin concentration by $y(t)$. Then
%
\begin{align*}
    \dot{n} &= (\alpha - \beta - K y) n, \\
    \dot{y} &= \gamma n - \delta y,
\end{align*}
%
with $\alpha, \beta, \gamma, \delta, K \geq 0$.

(b) Find the nullclines, the steady states, and sketch a phase portrait.

\textit{Solution.}

\vspace{5mm}

(c) Sketch the vector field.

\textit{Solution.}

\vspace{5mm}

(d) Linearize the system and characterize each of the steady states
(stable/unstable, saddle, node, spiral, center, etc.). Find the regions
in parameter space such that the nontrivial (coexistence) equilibrium is
either a node or a spiral.

\textit{Solution.}

\vspace{5mm}

(e) Sketch some trajectories for the case of $\delta < 4(\alpha - \beta)$,
and explain what you see in terms of the biology.

\textit{Solution.}

\newpage

3.9.17. Imagine a small pond that is mature enough to support wildlife.
We desire to stock the pond with game fish, say trout or bass Let $T(t)$
denote the population of the trout at any time $t$, and let $B(t)$
denote the bass population.

(a) Initially, assume that the pond environment can support an unlimited
number of trout in isolation (i.e., growth of the trout population is
exponential). Write down an equation that describes the evolution of the
trout population in the absence of competition.

\textit{Solution.}

\vspace{5mm}

(b) Modify the equation to account for competition of the trout with the
bass population for living space and a common food supply. You may assume
that the growth rate of the trout population depends linearly on the bass
population.

\textit{Solution.}

\vspace{5mm}

(c) Repeat (a) and (b) for the bass population.

\textit{Solution.}

\vspace{5mm}

(e) What are the steady states of the system? Determine stability of the
steady states using linearization.

\textit{Solution.}

\end{document}
