% Set up the document
\documentclass{article}

% Page size
\usepackage[
    letterpaper,]{geometry}

% Lines between paragraphs
\setlength{\parskip}{\baselineskip}
\setlength{\parindent}{0pt}

% Math
\usepackage{mathtools}
\usepackage{amssymb}
\usepackage{amsthm}
\usepackage{commath}

% Number sets
\newcommand{\C}{\mathbb{C}}
\newcommand{\N}{\mathbb{N}}
\newcommand{\Q}{\mathbb{Q}}
\newcommand{\R}{\mathbb{R}}
\newcommand{\Z}{\mathbb{Z}}

% Links
\usepackage{hyperref}

% Page numbers at top right
\usepackage{fancyhdr}
\pagestyle{fancy}
\fancyhf{}
\fancyhead[R]{\thepage}
\renewcommand\headrulewidth{0pt}

% Graphics
\usepackage{float}
\usepackage{graphicx}
\graphicspath{ {./img/} }

\begin{document}

\textbf{MATH 461 assignment 4} \\
\textbf{Matt Wiens \#301294492} \\
\textbf{2020-07-03}

3.9.2. Psychologists interested in learning theory study learning
curves. A learning curve is a graph of a function $P(t)$, the
performance of someone learning a skill as a function of the training
time $t$.

(a) What does $\od{P}{t}$ represent?

\textit{Solution.}
The derivative $\od{P}{t}$ represents the rate at which an individual is
learning at time $t$ into their training time.

\vspace{5mm}

(b) Discuss why the differential equation
%
\begin{equation*}
    \dod{P}{t} = k (M - P),
\end{equation*}
%
where $k$ and $M$ are positive constants, is a reasonable model for
learning. What is the meaning of $k$ and $M$? What would be a reasonable
initial condition for the model? Include a graph of $\od{P}{t}$ versus
$P$ as part of your discussion.

\textit{Solution.}
Qualitatively, for $P \leq M$ the $M - P$ term says that the rate at
which you learn decreases the more you learn (i.e., the better your
performance), up to some maximum performance level $M$, after which you
stop learning. This is fairly intuitive for me.

For $P > M$, this term says that your performance will decay down to $M$
at a rate proportional to the difference between your performance and the
``maximum'' performance level. I find this case quite a bit less intuitive
than $P \leq M$, but it is a reasonable assumption that $P \leq M$ whenever
this is applied, allowing us to ignore this case.

The constant $k$ essentially scales the $M - P$ term, and so governs how
rapidly one learns (or decays for $P > M$) at any given performance level $P$.

Based on my comments so far, I would say that any initial condition
with $P \leq M$ is reasonable for this model. $P > M$ could work too,
with proper justification in the particular application of this model.

A plot of the differential equation is shown, with the particular case
of $k = 1/3$, $M = 100$, in Figure~\ref{fig:q392a}.

\begin{figure}[!ht]
    \includegraphics[width=5in]{q392a}
    \centering
    \caption{Plot of $\od{P}{t}$ versus $P$ for $k = 1/3$, $M = 100$}
    \label{fig:q392a}
\end{figure}

\vspace{5mm}

(c) Make a qualitative sketch of solutions to the differential equation.

\textit{Solution.}
Noting that
%
\begin{equation*}
    \dod{(M - P)}{t} = - \dod{P}{t} = - k (M - P)
\end{equation*}
%
we see that the change in learning decays down to zero exponentially in time.
This gives us enough to qualitatively sketch the solutions. However,
solving this system is very easy, so I'm going to go ahead and do it.
That is, the solutions are of the form
%
\begin{equation*}
    M - P = c e^{-k t} \implies P = M - c e^{-k t},
\end{equation*}
%
where the $c$ is related to the initial performance $P_0$ by
$c = M - P_0$. Hence our solutions are given by
%
\begin{equation*}
    P = M - (M - P_0) e^{-k t}
    .
\end{equation*}
%
The solution is plotted below in Figure~\ref{fig:q392b}, taking
different values of $k$ and fixing $P_0 = 0$, $M = 100$. (Varying $P_0$
isn't very insightful since this just amounts to horizontally
translating the solution curves.)

\begin{figure}
    \includegraphics[width=5in]{q392b}
    \centering
    \caption{Plots of $P(t)$ for $P_0 = 0$, $M = 100$}
    \label{fig:q392b}
\end{figure}

\newpage

3.9.16. Some populations produce waste products, which in high concentrations are
toxic to the population itself. For example, algae or bacteria show the structure
in Figure 3.25 (in the course textbook). Let the population density be denoted
by $n(t)$ and the toxin concentration by $y(t)$. Then
%
\begin{align*}
    \dot{n} &= (\alpha - \beta - K y) n, \\
    \dot{y} &= \gamma n - \delta y,
\end{align*}
%
with $\alpha, \beta, \gamma, \delta, K \geq 0$.

(b) Find the nullclines and the steady states.

\textit{Solution.}
Following the notation in the textbook, let
%
\begin{align*}
    f_1(n, y) &= (\alpha - \beta - K y) n, \\
    f_2(n, y) &= \gamma n - \delta y
    .
\end{align*}
%
The $n$-nullcline is then the set of all points $(n, y)$ such that
$f_1(n, y) = 0$, which we can see by inspection is given by
%
\begin{equation*}
    \cbr{(0, y): y \geq 0} \cup \cbr{\del{n, \frac{\alpha - \beta}{K}}: n \geq 0}
    .
\end{equation*}
%
The $y$-nullcline, on the other hand, is the set of all points $(n, y)$
such that $f_2(n, y) = 0$. This is given by (again by inspection)
%
\begin{equation*}
    \cbr{\del{\frac{\delta y}{\gamma}, y}: y \geq 0}
    .
\end{equation*}
%
We can find the steady states by taking the intersection of the $n$- and
$y$-nullclines, but this system has so many parameters that its easier
just to use Maple to solve
%
\begin{align*}
    0 &= (\alpha - \beta - K y) n, \\
    0 &= \gamma n - \delta y.
\end{align*}
%
Solving the above system of equations in Maple we obtain the steady state
solutions
%
\begin{equation*}
    (n, y) \in \cbr{
        (0, 0),
        \del{
            \frac{\delta (\alpha - \beta)}{\gamma K},
            \frac{\alpha - \beta}{K}
        }}
        .
\end{equation*}

\vspace{5mm}

(d) Linearize the system and characterize each of the steady states
(stable/unstable, saddle, node, spiral, center, etc.). Find the regions
in parameter space such that the nontrivial (coexistence) equilibrium is
either a node or a spiral.

\textit{Solution.}
We start by computing the Jacobian matrix for our system:
%
\begin{equation*}
    J =
    \begin{bmatrix}
        \partial_n f_1 & \partial_y f_1 \\
        \partial_n f_2 & \partial_y f_2
    \end{bmatrix}
    =
    \begin{bmatrix}
        \alpha - \beta - K y & - K n \\
        \gamma & - \delta
    \end{bmatrix}
\end{equation*}
%
Evaluating this matrix at each of our steady states we have
(letting $J(n, y)$ denote the Jacobian evaluated at the point
$(n, y)$)
%
\begin{align*}
    J(0, 0) &=
    \begin{bmatrix}
        \alpha - \beta & 0 \\
        \gamma & - \delta
    \end{bmatrix}
    \eqqcolon J_0
    ,
    \\
    J \del{
            \frac{\delta (\alpha - \beta)}{\gamma K},
            \frac{\alpha - \beta}{K}
        }
    &=
    \begin{bmatrix}
        0 & - \frac{\delta (\alpha - \beta)}{\gamma} \\
        \gamma & - \delta
    \end{bmatrix}
    \eqqcolon J_1
    .
\end{align*}
%
For $J_0$ we can see that the eigenvalues are simply the diagonal elements,
that is $\lambda_1 = \alpha - \beta$, $\lambda_2 = - \delta$. Since we
know that $\delta \geq 0$, either $(0, 0)$ is a saddle or an unstable node,
depending on whether $\lambda_1 < 0$ or not, respectively.

For $J_1$ we compute the characteristic polynomial and set it to zero:
%
\begin{equation*}
    \det(J_1 - \lambda I) = \lambda^2 + \delta \lambda + \delta (\alpha - \beta) = 0
    .
\end{equation*}
%
This has solutions
%
\begin{equation*}
    \lambda = - \frac{\delta}{2} \pm \frac{1}{2} \sqrt{\delta^2 - 4(\alpha - \beta)}
    .
\end{equation*}
%
This will be a stable node provided that
%
\begin{equation*}
    \delta^2 \geq 4 (\alpha - \beta)
\end{equation*}
%
(note that both eigenvalues will be real and negative in this case); otherwise
the coexistence equilibrium will be a stable spiral.

\vspace{5mm}

(e) Sketch some trajectories for the case of $\delta < 4(\alpha - \beta)$,
and explain what you see in terms of the biology.

\textit{Solution.}
In order to be able to plot anything, we need to fix the values of our
constants. Let $\delta = 1 / 2$, $\alpha = 1$, $\beta = 1/2$,
$\gamma = 1$, $K = 1$.

Using these values, we have the vector field shown in
Figure~\ref{fig:q3916a} along with some solutions plotted
in~\ref{fig:q3916b}. We can see clearly that the coexistence equilibrium
is in fact stable. In terms of the biology, it makes sense that we see
spiral like behaviour for similar reasons to the predatory prey model:
when the toxin concentration is low, the population can flourish; the
higher the population the higher the toxin concentration, which
eventually leads to a high enough toxin concentration such that the
population dies out; but as the population dies out then the toxin
concentration lowers. In this particular case, there is a stable
coexistence equilibrium that is reached quickly be most initial
conditions.

\begin{figure}
    \includegraphics[width=4in]{q3916a}
    \centering
    \caption{Vector field using $\delta = 1 / 2$, $\alpha = 1$, $\beta = 1/2$ $\gamma = 1$, $K = 1$}
    \label{fig:q3916a}
\end{figure}

\begin{figure}
    \includegraphics[width=4in]{q3916b}
    \centering
    \caption{Vector field with particular solutions using $\delta = 1 / 2$, $\alpha = 1$, $\beta = 1/2$ $\gamma = 1$, $K = 1$}
    \label{fig:q3916b}
\end{figure}

\newpage

3.9.17. Imagine a small pond that is mature enough to support wildlife.
We desire to stock the pond with game fish, say trout or bass. Let $T(t)$
denote the population of the trout at any time $t$, and let $B(t)$
denote the bass population.

(a) Initially, assume that the pond environment can support an unlimited
number of trout in isolation (i.e., growth of the trout population is
exponential). Write down an equation that describes the evolution of the
trout population in the absence of competition.

\textit{Solution.}
If there are no limits to the growth of the trout, then
the trout population grows in proportion to its size at any time $t$.
That is,
%
\begin{equation*}
    \dod{T}{t} = k_1 T
    ,
\end{equation*}
%
where $k_1 > 0$.

\vspace{5mm}

(b) Modify the equation to account for competition of the trout with the
bass population for living space and a common food supply. You may assume
that the growth rate of the trout population depends linearly on the bass
population.

\textit{Solution.}
Now we need to account for competition. I would naturally assume we
need to account for interactions (and thus have $T B$ terms in the differential
equation) but the question tells us the growth has a linear dependence on the
bass population, so we have
%
\begin{equation*}
    \dod{T}{t} = k_1 T - k_2 B
    ,
\end{equation*}
%
where $k_2 > 0$.

\vspace{5mm}

(c) Repeat (a) and (b) for the bass population.

\textit{Solution.}
The equivalent equation for part (a) for the bass population is
%
\begin{equation*}
    \dod{B}{t} = k_3 B
    ,
\end{equation*}
%
where $k_3 > 0$; and the equivalent equation for part (b) for the
bass population is
%
\begin{equation*}
    \dod{B}{t} = k_3 B - k_4 T
    ,
\end{equation*}
%
where $k_4 > 0$.

\vspace{5mm}

(e) What are the steady states of the system? Determine stability of the
steady states using linearization.

\textit{Solution.}
Now we consider the system
%
\begin{align*}
    \dod{T}{t} &= k_1 T - k_2 B, \\
    \dod{B}{t} &= k_3 B - k_4 T
    .
\end{align*}
%
Solving for the steady states in
%
\begin{align*}
    0 &= k_1 T - k_2 B, \\
    0 &= k_3 B - k_4 T
    ,
\end{align*}
%
clearly we have the trivial equilibrium $(0, 0)$. By performing some simple
algebra, there is an additional solution provided that
%
\begin{equation*}
    k_1 = \frac{k_2 k_4}{k_3}
    ;
\end{equation*}
%
if this condition is true then $(T, k_4 / k_3 \cdot T)$, where $T \geq 0$, is
are coexistence equilibriums.

To determine stability, let
%
\begin{align*}
    f_1(T, B) &= k_1 T - k_2 B, \\
    f_2(T, B) &= -k_4 T + k_3 B
    .
\end{align*}
%
We evaluate the Jacobian
%
\begin{equation*}
    J =
    \begin{bmatrix}
        \partial_T f_1 & \partial_B f_1 \\
        \partial_T f_2 & \partial_B f_2
    \end{bmatrix}
    =
    \begin{bmatrix}
        k_1 & -k_2 \\
        -k_4 & k_3
    \end{bmatrix}
    .
\end{equation*}
%
To now solve for the eigenvalues:
%
\begin{align*}
    &\det(J - \lambda I) = 0 \\
    &\implies \lambda^2 + (-k_1 - k_3) \lambda + (k_1 k_3 - k_2 k_4) = 0 \\
    &\implies \lambda = \frac{k_1 + k_3}{2} \pm \frac{1}{2} \sqrt{(k_1 + k_3)^2 - 4 (k_1 k_3 - k_2 k_4)}
    .
\end{align*}
%
We see that both eigenvalues have positive real parts, so all equilibria
are unstable.

\end{document}
