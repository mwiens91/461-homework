% Set up the document
\documentclass{article}

% Page size
\usepackage[
    letterpaper,]{geometry}

% Lines between paragraphs
\setlength{\parskip}{\baselineskip}
\setlength{\parindent}{0pt}

% Math
\usepackage{mathtools}
\usepackage{amssymb}
\usepackage{amsthm}
\usepackage{commath}

% Number sets
\newcommand{\C}{\mathbb{C}}
\newcommand{\N}{\mathbb{N}}
\newcommand{\Q}{\mathbb{Q}}
\newcommand{\R}{\mathbb{R}}
\newcommand{\Z}{\mathbb{Z}}

% Links
\usepackage{hyperref}

% Page numbers at top right
\usepackage{fancyhdr}
\pagestyle{fancy}
\fancyhf{}
\fancyhead[R]{\thepage}
\renewcommand\headrulewidth{0pt}

% Graphics
\usepackage{float}
\usepackage{graphicx}
\graphicspath{ {./img/} }

\begin{document}

\textbf{MATH 461 assignment 3} \\
\textbf{Matt Wiens \#301294492} \\
\textbf{2020-06-19}

3.9.4. In this exercise, you will be considering three simple models of
a fishery. Let $N(t)$ be the population of fish at time $t$. In the
absence of fishing, the population is assumed to grow logistically,
that is,
%
\begin{equation*}
    \dot N = r N \del{1 - \frac{N}{K}}
    ,
\end{equation*}
%
where $r > 0$ is the intrinsic growth of the population, and $K >0$ is
the carrying capacity for the fish population. The effects of fishing
are modeled with an additional term in the equation for $N$. The three
models are as follows:
%
\begin{align*}
    \text{\textbf{Model 1:}} \ \dot N &= r N \del{1 - \frac{N}{K}} - H_1; \\
    \text{\textbf{Model 2:}} \ \dot N &= r N \del{1 - \frac{N}{K}} - H_2 N; \\
    \text{\textbf{Model 3:}} \ \dot N &= r N \del{1 - \frac{N}{K}} - H_3 \frac{A}{A + N},
\end{align*}
%
where $H_1$, $H_2$, $H_3$, and $A$ are positive constants.

(a) For each model, give a biological interpretation of the fishing
term. How do they differ? What is the meaning of the constants $H_1$,
$H_2$, $H_3$, and $A$?

\textit{Solution.}

\vspace{5mm}

(b) Critique model 1. Why is it not biologically realistic?

\textit{Solution.}

\vspace{5mm}

(c) Which of Models 2 or 3 do you think is best and why?

\textit{Solution.}

\newpage

3.9.5. Levins suggested modeling not the number of individuals but the
fraction of patches that a population occupies. He suggested the
following equation:
%
\begin{equation*}
    P^\prime = c P (h - P) - \mu P
    ,
\end{equation*}
%
where $P(t)$ denotes the fraction of occupied patches. The number $h$
denotes the fraction of patches that is actually habitable for the
population and hence $h - P$ is the number of empty but habitable
patches. Note that $0 \leq P \leq h \leq 1$. The population colonizes
empty patches with rate $c$. Occupied patches become empty with rate
$\mu$.

(a) Find the steady states of the system.

\textit{Solution.}

\vspace{5mm}

(b) Assume that $h$ can be varied (e.g., construction takes up habitable
patches). Draw the bifurcation diagram with $h$ as the parameter. Do all
the habitable patches have to be destroyed before the population dies
out?

\textit{Solution.}

\newpage

3.9.6. Consider a gene that is activated by the presence of a
biochemical substance $S$. Let $g(t)$ denote the concentration of the
gene product at time $t$, and assume that the concentration of $S$,
denoted by $s_0$, is fixed. A model describing the dynamics of $g$
is as follows:
%
\begin{equation}
    \dod{g}{t} = k_1 s_0 - k_2 g + \frac{k_3 g^2}{k_4^2 + g^2}
    \label{eq:396-1}
    ,
\end{equation}
%
where the $k$'s are positive constants.

(b) Show that~\eqref{eq:396-1} can be put in the dimensionless form
%
\begin{equation*}
    \dod{x}{\tau} = s - r x + \frac{x^2}{1 + x^2}
    ,
\end{equation*}
%
where $r > 0$ and $s \geq 0$ are dimensionless groups. What are $r$ and
$s$ in terms of the original model parameters?

\textit{Solution.}

\newpage

3.9.7. We study $2 \times 2$ systems of linear ODEs:
%
\begin{equation*}
    y^\prime = A y, \quad
    y =
    \begin{pmatrix}
        y_1 \\
        y_2
    \end{pmatrix}, \quad
    A =
    \begin{pmatrix}
        a & b \\
        c & d
    \end{pmatrix}
    .
\end{equation*}
%
Classify the origin $\begin{pmatrix}0 \\ 0\end{pmatrix}$ as either
stable/unstable spiral, node, or saddle, and plot (or sketch) the
phase portrait for each of the following cases:
%
\begin{equation*}
    A =
    \begin{pmatrix}
        1 & 1 \\
        3 & -1
    \end{pmatrix}
    ,
    \begin{pmatrix}
        2 & 1 \\
        2 & 3
    \end{pmatrix}
    ,
    \begin{pmatrix}
        -1 & -2 \\
        2 & -1
    \end{pmatrix}
    .
\end{equation*}

\textit{Solution.}

\end{document}
