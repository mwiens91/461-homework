% Set up the document
\documentclass{article}

% Page size
\usepackage[
    letterpaper,]{geometry}

% Lines between paragraphs
\setlength{\parskip}{\baselineskip}
\setlength{\parindent}{0pt}

% Math
\usepackage{mathtools}
\usepackage{amssymb}
\usepackage{amsthm}
\usepackage{commath}

% Number sets
\newcommand{\C}{\mathcal{C}}
\newcommand{\N}{\mathbb{N}}
\newcommand{\Q}{\mathbb{Q}}
\newcommand{\R}{\mathbb{R}}
\newcommand{\Z}{\mathbb{Z}}

% Links
\usepackage{hyperref}

% Page numbers at top right
\usepackage{fancyhdr}
\pagestyle{fancy}
\fancyhf{}
\fancyhead[R]{\thepage}
\renewcommand\headrulewidth{0pt}

\begin{document}

\textbf{MATH 342 assignment 1} \\
\textbf{Matt Wiens \#301294492} \\
\textbf{2020-05-22}

1.4.1. Assume you have a culture of bacteria crowing in a petri dish,
and each cell divides into two identical copies of itself every 10
minutes.

(a) Choose a unit time, and find the corresponding probability of cell division.

\textit{Solution.}

\newpage

(b) Write down a discrete-time model which balances the amount of cells
at time $t$ and at time $t + \Delta t$.

\textit{Solution.}

\newpage

(c) Define the growth rate, and derive the corresponding continuous-time model.

\textit{Solution.}

\newpage

(d) Solve both the discrete-time model and continuous-time models, and
compare the solutions.

\textit{Solution.}

\newpage

(e) When is a discrete-time model appropriate? When is a continuous-time
model appropriate?

\textit{Solution.}

\newpage

1.4.2. Study the two models (1.2), (1.1) which lead to Figure 1.2 (all
  numbered references are referring to the course textbook) and vary the
  time increment $\Delta t$ (e.g., try $\Delta t = \frac{1}{4} \text{
  day}$, $\frac{1}{8} \text{ day}$, $1 \text{ day}$, $2 \text{ days}$,
  $10 \text{ days}$). What do you observe? Which choice of $\Delta t$
  gives the best, and which gives the worst agreement? Can you explain
  why?

\textit{Solution.}

\end{document}
