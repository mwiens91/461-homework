% Set up the document
\documentclass{article}

% Page size
\usepackage[
    letterpaper,]{geometry}

% Lines between paragraphs
\setlength{\parskip}{\baselineskip}
\setlength{\parindent}{0pt}

% Math
\usepackage{mathtools}
\usepackage{amssymb}
\usepackage{amsthm}
\usepackage{commath}

% Number sets
\newcommand{\C}{\mathbb{C}}
\newcommand{\N}{\mathbb{N}}
\newcommand{\Q}{\mathbb{Q}}
\newcommand{\R}{\mathbb{R}}
\newcommand{\Z}{\mathbb{Z}}

% Links
\usepackage{hyperref}

% Page numbers at top right
\usepackage{fancyhdr}
\pagestyle{fancy}
\fancyhf{}
\fancyhead[R]{\thepage}
\renewcommand\headrulewidth{0pt}

% Graphics
\usepackage{float}
\usepackage{graphicx}
\graphicspath{ {./img/} }

\begin{document}

\textbf{MATH 461 assignment 5} \\
\textbf{Matt Wiens \#301294492} \\
\textbf{2020-07-31}

3.9.14. Write down a differential equation for the following pathway:
%
\begin{equation*}
    A \xrightleftharpoons[k_1]{k_{-1}}
    B \xrightleftharpoons[k_2]{k_{-2}}
    C \xrightleftharpoons[k_3]{k_{-3}}
    A
    .
\end{equation*}

\textit{Solution.}

\newpage

4.5.2. (a) Show that the function
%
\begin{equation*}
    g(x, t) = \frac{1}{2 \sqrt{\pi D t}} \exp \del[3]{\frac{x^2}{4 D t}}
\end{equation*}
%
solves the diffusion equation $u_t = D u_{x x}$.

\textit{Solution.}

\vspace{5mm}

(b) Make sure that $g(x, t) \geq 0$ for all $t \geq 0$ and $x \in \R$ and
investigate the limits of $x \to \pm \infty$ and $t \to \infty$.

\textit{Solution.}

\newpage

4.5.3. Certain ant species (such as \textit{Pogonomyrmex badius}) use
pheromones as a signal for danger. A good model for the spread of the
pheromones in the tube is the one-dimensional diffusion equation. In
experiments, Bosseri and Wilson released ants in a long tube and stimulated
one ant until it released a pheromone. They measured within which distance
and after which time delay the other ants would react to the signal.
We assume that at time $t = 0$ a signal of strength $\alpha$ is released.
The diffusion constant is $D = 1$. Other ants react to the stimulus if the
concentration they perceive is $10\%$ of $\alpha$ or higher.

(a) For each $t > 0$, find the region in the tube $0 \leq x \leq x(t)$ where
the ants would react to the stimulus (region of influence).

\textit{Solution.}

\vspace{5mm}

(b) Sketch the time evolution of $x(t)$.

\textit{Solution.}

\vspace{5mm}

(c) Find the time $t^*$ such that the region of influence is empty for all
$t > t^*$.

\textit{Solution.}

\newpage

5.8.2. Use Table 5.2 (from the course textbook) and show that the eigenvector
$u^*$ of the transition matrix $P$ with eigenvalue $1$ is given by formula 5.3
(also from the course textbook).

\textit{Solution.}

\end{document}
