% Set up the document
\documentclass{article}

% Page size
\usepackage[
    letterpaper,]{geometry}

% Lines between paragraphs
\setlength{\parskip}{\baselineskip}
\setlength{\parindent}{0pt}

% Math
\usepackage{mathtools}
\usepackage{amssymb}
\usepackage{amsthm}
\usepackage{commath}

% Number sets
\newcommand{\C}{\mathbb{C}}
\newcommand{\N}{\mathbb{N}}
\newcommand{\Q}{\mathbb{Q}}
\newcommand{\R}{\mathbb{R}}
\newcommand{\Z}{\mathbb{Z}}

% Links
\usepackage{hyperref}

% Page numbers at top right
\usepackage{fancyhdr}
\pagestyle{fancy}
\fancyhf{}
\fancyhead[R]{\thepage}
\renewcommand\headrulewidth{0pt}

% Graphics
\usepackage{float}
\usepackage{graphicx}
\graphicspath{ {./img/} }

\begin{document}

\textbf{MATH 461 assignment 5} \\
\textbf{Matt Wiens \#301294492} \\
\textbf{2020-07-31}

3.9.14. Write down a differential equation for the following pathway:
%
\begin{equation*}
    A \xrightleftharpoons[k_1]{k_{-1}}
    B \xrightleftharpoons[k_2]{k_{-2}}
    C \xrightleftharpoons[k_3]{k_{-3}}
    A
    .
\end{equation*}

\textit{Solution.}
For this pathway we have the following system of differential equations:
%
\begin{align*}
    \dod{a}{t} &= - k_{-1} a + k_1 b + k_{-3} c - k_3 a
    , \\
    \dod{b}{t} &= k_{-1} a - k_1 b - k_{-2} b + k_2 c
    , \\
    \dod{c}{t} &= k_{-2} b - k_2 c - k_{-3} c + k_3 a
    ,
\end{align*}
%
where $a$, $b$, and $c$, denote the concentrations of A, B, and C, respectively.
Combining terms, we can equivalently write the system as
%
\begin{align*}
    \dod{a}{t} &= - (k_{-1} + k_3) a + k_1 b + k_{-3} c
    , \\
    \dod{b}{t} &= k_{-1} a - (k_1 + k_{-2}) b + k_2 c
    , \\
    \dod{c}{t} &= k_3 a + k_{-2} b - (k_2 + k_{-3}) c
    .
\end{align*}

\newpage

4.5.2. (a) Show that the function
%
\begin{equation*}
    g(x, t) = \frac{1}{2 \sqrt{\pi D t}} \exp \del[3]{- \frac{x^2}{4 D t}}
\end{equation*}
%
solves the diffusion equation $u_t = D u_{x x}$.

\textit{Solution.}
Throughout this solution, note that we will make repeated use of the chain rule.
Calculating $g_t$ we have
%
\begin{align*}
    \dpd{g}{t}
        &=
        \dpd{}{t} \del{\frac{1}{2 \sqrt{\pi D t}}} \exp \del[3]{- \frac{x^2}{4 D t}}
        +
        \frac{1}{2 \sqrt{\pi D t}} \dpd{}{t} \del{\exp \del[3]{- \frac{x^2}{4 D t}}}
        \\
        &=
        -
        \frac{1}{2 t} \cdot \frac{1}{2 \sqrt{\pi D t}} \exp \del[3]{- \frac{x^2}{4 D t}}
        +
        \frac{1}{2 \sqrt{\pi D t}} \cdot \frac{x^2}{4D} \cdot \frac{1}{t^2} \cdot \exp \del[3]{- \frac{x^2}{4 D t}}
        \\
        &=
        \del{- \frac{1}{2 t} + \frac{x^2}{4Dt^2}} g(x, t)
        .
\end{align*}
%
For the $x$-derivatives, we first calculate $g_x$:
%
\begin{align*}
    \dpd{g}{x}
        &= \frac{1}{2 \sqrt{\pi D t}} \dpd{}{x} \del{\exp \del[3]{- \frac{x^2}{4 D t}}}
        \\
        &= - \frac{1}{2 \sqrt{\pi D t}} \cdot \frac{1}{4Dt} \cdot 2 x \cdot \exp \del[3]{- \frac{x^2}{4 D t}}
        \\
        &= - \frac{x}{2Dt} g(x, t)
        .
\end{align*}
%
Using our expression for $g_x$, we can calculate $g_{xx}$:
%
\begin{align*}
    \dpd[2]{g}{x}
        &= \dpd{}{x} \del{\dpd{g}{x}}
        \\
        &= - \dpd{}{x} \del{\frac{x}{2Dt} g(x, t)}
        \\
        &=
        -
        \dpd{}{x} \del{\frac{x}{2Dt}} g(x, t)
        -
        \frac{x}{2Dt} \dpd{g}{x}
        \\
        &=
        -
        \frac{1}{2Dt} \cdot g(x, t)
        +
        \frac{x}{2Dt} \del{\frac{x}{2Dt} g(x, t)}
        \\
        &=
        \del{- \frac{1}{2 D t} + \frac{x^2}{4 D^2 t^2}} g(x, t)
        \\
        &=
        \frac{1}{D} \del{- \frac{1}{2 t} + \frac{x^2}{4 D t^2}} g(x, t)
        \\
        &=
        \frac{1}{D} g_t
        .
\end{align*}
%
Thus we have that $g_t = D g_{xx}$, and so $g$ solves the diffusion equation.

\vspace{5mm}

(b) Make sure that $g(x, t) \geq 0$ for all $t \geq 0$ and $x \in \R$ and
investigate the limits of $x \to \pm \infty$ and $t \to \infty$.

\textit{Solution.}
For $t > 0$ the term $\frac{1}{2 \sqrt{\pi D t}}$ is strictly positive.
The exponential function is also strictly positive. Hence $g(x, t) > 0$
(and, by extension, $g(x, t) \geq 0$) for all $t > 0$ and $x \in \R$.
What happens when $t = 0$? Strictly speaking, $g(x, 0)$ is not defined,
but we can investigate the ``right-handed'' limit
%
\begin{equation*}
    \lim_{t \to 0^+} g(x, t)
    .
\end{equation*}
%
This is a non-trivial calculation, so I will simply state the result
that $g(x, t) \to \delta_0$ as $t \to 0^+$ in the sense of distributions,
where $\delta_0$ is the Dirac delta function (which is technically a
distribution).

As $t \to \infty$ we have $\frac{1}{2 \sqrt{\pi D t}} \to 0$ and also
$\exp ( - \frac{x^2}{4 D t} ) \to 1$ for all $x \in \R$, and
hence $g(x, t) \to 0$ for all $x \in \R$.

In the spatial domain, for any $t > 0$ we have that
$\exp ( - \frac{x^2}{4 D t} ) \to 0$ as $x \to \pm \infty$, so
$g(x, t) \to 0$ as $x \to \pm \infty$ for any $t > 0$.


\newpage

4.5.3. Certain ant species (such as \textit{Pogonomyrmex badius}) use
pheromones as a signal for danger. A good model for the spread of the
pheromones in the tube is the one-dimensional diffusion equation. In
experiments, Bosseri and Wilson released ants in a long tube and stimulated
one ant until it released a pheromone. They measured within which distance
and after which time delay the other ants would react to the signal.
We assume that at time $t = 0$ a signal of strength $\alpha$ is released.
The diffusion constant is $D = 1$. Other ants react to the stimulus if the
concentration they perceive is $10\%$ of $\alpha$ or higher.

(a) For each $t > 0$, find the region in the tube $0 \leq x \leq x(t)$ where
the ants would react to the stimulus (region of influence).

\textit{Solution.}

\vspace{5mm}

(b) Sketch the time evolution of $x(t)$.

\textit{Solution.}

\vspace{5mm}

(c) Find the time $t^*$ such that the region of influence is empty for all
$t > t^*$.

\textit{Solution.}

\newpage

5.8.2. Use Table 5.2 (from the course textbook) and show that the eigenvector
$u^*$ of the transition matrix $P$ with eigenvalue $1$ is given by formula 5.3
(also from the course textbook).

\textit{Solution.}

\end{document}
