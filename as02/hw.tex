% Set up the document
\documentclass{article}

% Page size
\usepackage[
    letterpaper,]{geometry}

% Lines between paragraphs
\setlength{\parskip}{\baselineskip}
\setlength{\parindent}{0pt}

% Math
\usepackage{mathtools}
\usepackage{amssymb}
\usepackage{amsthm}
\usepackage{commath}

\DeclareMathOperator{\tr}{tr}

% Number sets
\newcommand{\C}{\mathcal{C}}
\newcommand{\N}{\mathbb{N}}
\newcommand{\Q}{\mathbb{Q}}
\newcommand{\R}{\mathbb{R}}
\newcommand{\Z}{\mathbb{Z}}
\newcommand{\as}{a^*}
\newcommand{\xs}{x^*}
\newcommand{\As}{A^*}
\newcommand{\Bs}{B^*}

% Links
\usepackage{hyperref}

% Page numbers at top right
\usepackage{fancyhdr}
\pagestyle{fancy}
\fancyhf{}
\fancyhead[R]{\thepage}
\renewcommand\headrulewidth{0pt}

% Graphics
\usepackage{float}
\usepackage{graphicx}
\graphicspath{ {./img/} }

\begin{document}

\textbf{MATH 461 assignment 2} \\
\textbf{Matt Wiens \#301294492} \\
\textbf{2020-06-05}

2.4.2. Consider the following model for a drug prescription:
%
\begin{equation*}
    a_{n + 1} = a_n - k a_n + b,
\end{equation*}
%
where $a_n$ is the amount of a drug (in mg, say) in the bloodstream
after administration of $n$ dosages at regular intervals (hourly, say).

(a) Discuss the meaning of the model parameters $k$ and $b$. What can
you say about their size and sign?

\textit{Solution.}
My interpretation of $k$ and $b$ is that $k$ is the average percentage of the
drug that is removed from the bloodstream (through the kidneys for
example) during each interval, while $b$ is the amount (in mg) of the net dosage added to the
bloodstream during each interval. Based on my interpretations, I assume
that $k \in (0, 1]$ and that $b > 0$ with $b$ in size being around the
same magnitude as the $a_n$s.

\vspace{5mm}

(b) Find the fixed points of the model and their stability via linearization.

\textit{Solution.}
Let $\as$ represent a fixed point of this model. Then we have
%
\begin{align*}
    \as = \as - k \as + b \\
    \implies \as = \frac{b}{k}
    ,
\end{align*}
%
which is a unique fixed point. Using linear stability analysis, we can
determine the stability of this fixed point by looking at the magnitude
of the iterating function's derivative at this point. For our model,
we have that $a_{n + 1} = f(a_n)$ where $f$ is given by
%
\begin{equation*}
    f(a) = a - k a + b
    .
\end{equation*}
%
Evaluating the derivative of this function we have that
%
\begin{equation*}
    f^\prime(a) = 1 - k
    .
\end{equation*}
%
By my assumption on $k$ we have that $|f^\prime(a)| < 1$ for all $a$, so
the fixed point we determined above is stable.

\vspace{5mm}

(c) Perform a cobwebbing analysis for this model. What happens to the
amount of drug in the bloodstream in the long run? How does the result
depend on the model parameters?

\textit{Solution.}
After experimenting with the applet Dr. Ruuth demonstrated during
lectures, the amount of drug in the bloodstream appears to always
approach the fixed point in the long run. This does not depend on the
model parameters. However, the speed at which this happens depends strongly
on $k$. Based on my observations, convergence takes a long time to occur
when $k$ is very close to $0$ (which makes sense given the formula for
the fixed point).

These results make sense given that there is exactly one fixed point and
that it is always stable.

\vspace{5mm}

(d) How should $b$ be chosen to ensure that the drug is effective, but
not toxic?

\textit{Solution.}
For most drugs you want the amount of it in the bloodstream to be
constant, so here we have that if $\as$ is the desired amount to be in
the bloodstream, that $b$ should be chosen so that
%
\begin{align*}
    \as = \as - k \as + b \\
    \implies b = k \as
    .
\end{align*}

\newpage

2.4.6. The exercise deals with the second-iterate map, $f^2(x)$, for the
  logistic map. $f(x) = r x (1 - x)$.

(a) Compute $f^2(x)$.

\textit{Solution.} The map second-iterate map $f^2$ is given by
%
\begin{equation*}
    f^2(x) = r (rx(1 - x)) (1 - (rx(1 - x))) = r^2 x (1 - x) (r x^2 - r x + 1)
    .
\end{equation*}

\vspace{5mm}

(b) Find the fixed points of $f^2(x)$. Verify that a nontrivial
$2$-cycle exists only for $r > 3$.

\textit{Solution.}
A fixed point $\xs$ of $f^2$ is a point such that $f^2(\xs) = \xs$; that
is,
%
\begin{equation*}
    \xs = r^2 \xs (1 - \xs) (r (\xs)^2 - r \xs + 1)
    .
\end{equation*}
%
Moving all terms to one side and expanding, we have that
%
\begin{equation*}
    -r^3 (\xs)^4 + 2 r^3 (\xs)^3 + r^2(r + 1) (\xs)^2 + (r^2 - 1) \xs = 0
    .
\end{equation*}
%
This is complicated enough that it's easiest just to use Maple to solve
for the fixed points to obtain
%
\begin{equation*}
    \xs \in \cbr{
        0,
        \frac{r - 1}{r},
        \frac{r + 1 + \sqrt{r^2 - 2 r - 3}}{2 r},
        \frac{r + 1 - \sqrt{r^2 - 2 r - 3}}{2 r}
    }
    .
\end{equation*}
%
For the non-trivial 2-cycles, clearly $0$ is a trivial 2-cycle. We see
that for the first non-zero fixed point we require $r > 1$, while for
the remaining two we have that
%
\begin{equation*}
    r^2 - 2 r - 3 = (r + 1) (r - 3)
    ,
\end{equation*}
%
so in order for these to be valid points we require $r \geq 3$. Next,
we will show that for non-trivial 2-cycles we must have $r > 3$; that is,
we need to show $r = 3$ results in trivial 2-cycles.

By plugging in $r = 3$ our set of fixed points becomes $\xs \in \cbr{0,
\frac{2}{3}}$. We already pointed out that $0$ is a trivial 2-cycle.
Because $f(2/3) = 3 \cdot 2 / 3 (1 - 2/3) = 2/3$, it turns out that
$2/3$ is also a trivial 2-cycle. Hence non-trivial 2-cycles exist only
for $r > 3$ (if they exist at all).

By a similar calculation to what we did above, we can determine that
$(r - 1) / r$ is always a trivial 2-cycle. However we have that
%
\begin{align*}
    f\del{\frac{r + 1 + \sqrt{r^2 - 2 r - 3}}{2 r}}
        &= r \frac{r + 1 + \sqrt{r^2 - 2 r - 3}}{2 r} \del{1 - \frac{r + 1 + \sqrt{r^2 - 2 r - 3}}{2 r}} \\
        &= \frac{1}{4 r} \del{r + 1 + \sqrt{r^2 - 2 r - 3}} \del{r - 1 - \sqrt{r^2 - 2 r - 3}} \\
        &= \frac{1}{4 r} \del{2 r + 2 - 2 \sqrt{r^2 - 2 r - 3}} \\
        &= \frac{r + 1 - \sqrt{r^2 - 2 r - 3}}{2 r}
\end{align*}
%
and similarly
%
\begin{equation*}
    f\del{\frac{r + 1 - \sqrt{r^2 - 2 r - 3}}{2 r}}
        = \frac{r + 1 + \sqrt{r^2 - 2 r - 3}}{2 r}
    .
\end{equation*}
%
Since these points are never equal two each other for $r > 3$ this
establishes the existence of a non-trivial 2-cycle.

\vspace{5mm}

(c) Compute $\dod{}{x} f^2(x)$.

\textit{Solution.}
We can compute the derivative of $f^2$ using the product rule:
%
\begin{align*}
    \dod{f^2}{x}
        &= \dod{}{x} r^2 x (x - 1) (r x^2 - r x + 1) \\
        &= r^2 (1 - x) (r x^2 - r x + 1)
           - r^2 x (r x^2 - r x + 1)
           + r^2 x (1 - x) (2 r x - r) \\
        &=  r^2 (1 - 2 x) (2 r x^2 - 2 r x + 1)
        .
\end{align*}

\vspace{5mm}

(d) Verify that the nontrivial $2$-cycle is stable for $3 < r < 1 +
\sqrt{6}$, and unstable for $r > 1 + \sqrt{6}$.

\textit{Solution.}
Let's write the fixed points that result in the non-trivial 2-cycle as
%
\begin{equation*}
    \xs_t \coloneqq \frac{r + 1 + t \sqrt{r^2 - 2 r - 3}}{2 r}
\end{equation*}
%
where $t = 1, -1$. Substituting this point into the derivative of $f^2$
and simplifying (using Maple) we obtain that, for either value of $t$,
%
\begin{equation*}
    \partial x f^2(\xs_t) = -r^2 + 2 r + 4
    .
\end{equation*}
%
For $r > 3$, we have that $\partial x f^2(\xs_t) < 1$ and is a strictly
decreasing function. To find out when this polynomial (in $r$) becomes
$-1$ we need to solve
%
\begin{equation*}
    -r^2 + 2 r + 4 = - 1
    ,
\end{equation*}
%
which, using the quadratic formula, gives us a positive root $1 + \sqrt{6}$.
Hence we have that for $3 < 1 + \sqrt{6}$, $-1< \partial x f^2(\xs_t) < 1$
and so the nontrivial 2-cycle is stable; for $r > 1 + \sqrt{6}$,
$\partial x f^2(\xs_t) < - 1$, so it is unstable.


\newpage

2.4.13. Consider the following simple competition model:
%
\begin{align*}
    A_{n + 1} &= \mu_1 A_n - \mu_3 A_n B_n, \\
    B_{n + 1} &= \mu_2 B_n - \mu_4 A_n B_n,
\end{align*}
%
where $\mu_1, \mu_2, \mu_3, \mu_4$ are positive constants.

(a) Find all fixed points.

\textit{Solution.}

Let $(\As, \Bs)$ represent the fixed points of the model. Thus we have
%
\begin{align*}
    \As &= \mu_1 \As - \mu_3 \As \Bs, \\
    \Bs &= \mu_2 \Bs - \mu_4 \As \Bs,
\end{align*}
%
Clearly $(\As, \Bs) = (0, 0)$ is a valid fixed point.
If $(\As, \Bs) \neq (0, 0)$ then we can simplify the above system of equations to
%
\begin{align*}
    0 &= \mu_1 - 1 - \mu_3 \Bs, \\
    0 &= \mu_2 - 1 - \mu_4 \As.
\end{align*}
%
Solving for $\As$ and $\Bs$ we have
%
\begin{align*}
    \As &= \frac{\mu_2 - 1}{\mu_4}, \\
    \Bs &= \frac{\mu_1 - 1}{\mu_3},
\end{align*}
%
which gives us our other solution.


\vspace{5mm}

(b) Determine the stability of the fixed points for the specific case
$\mu_1 = 1.2, \mu_2 = 1.3, \mu_3 = 0.001, \mu_4 = 0.002$.

\textit{Solution.}
Here the Jacobian is given by
%
\begin{equation*}
    J =
        \begin{pmatrix}
            \mu_1 - \mu_3 B & -\mu_3 A \\
            - \mu_4 B & \mu_2 - \mu_4 A
        \end{pmatrix}
\end{equation*}
%
(this is determined by the first equation given to us).
Taking the determinant and trace of this system we have that
%
\begin{align*}
    \det J &= (\mu_1 - \mu_3 B) (\mu_2 - \mu_4 A) - (- \mu_3 A) (- \mu_4 B) \\
           &= -A \mu_1 \mu_4 - B \mu_2 \mu_3 + \mu_1 \mu_2
           ,
           \\
    \tr J &= \mu_1 + \mu_2 - \mu_3 B - \mu_4 A
    .
\end{align*}
%
Recall that for two-dimensional systems, the Jury conditions
%
\begin{equation*}
    |\tr J| < 1 + \det J < 2
\end{equation*}
%
are necessary and sufficient for stability. Now using the values of $\mu$
as specified above we have that
%
\begin{align*}
    \det J &= - 0.0024 A - 0.0013 B + 1.56, \\
    \tr J &= 2.5 - 0.001 B - 0.002 A
    .
\end{align*}
%
For the fixed points $(\As, \Bs) = (0, 0)$ we have $\det J = 1.56$ and
$\tr J = 2.5$ so the Jury conditions are satisfied. Hence this pair of
fixed points is stable.

For the fixed points
%
\begin{align*}
    (\As, \Bs)
        &= \del{\frac{\mu_2 - 1}{\mu_4}, \frac{\mu_1 - 1}{\mu_3}} \\
        &= (150, 200)
\end{align*}
%
we have $\det J = 0.94$ and $\tr J = 2$, so in this case the Jury
conditions are not satisfied, so these fixed points are unstable.

\end{document}
